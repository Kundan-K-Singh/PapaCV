% This template is designed to offer an aesthetically pleasing resume that adheres to a formal and institutional tone, making it suitable for applications to companies and research centers requiring a high degree of professionalism. Navy blue has been chosen as the primary color to align with these objectives.
% The code is well-documented and annotated, allowing users to easily customize and modify it according to their needs. Please note that the template's content is meant to be humorous and should not be taken literally. We are grateful for your interest in using this template for your professional endeavors.
% Author: Christian Maria Giannetti

%----------------------------------------------------------------------------------------
%  Packages And Other Document Configurations
%----------------------------------------------------------------------------------------

\documentclass{resume} % Use the custom resume.cls style

% Document margins
\usepackage[left=0.75in,top=0.6in,right=0.75in,bottom=0.6in]{geometry}

% Color and hyperlink packages
\usepackage{xcolor}
\usepackage{hyperref}

% Footnote and margin adjustment packages
\usepackage{footnote}
\usepackage{changepage}

% Fontawesome package for icons
\usepackage{fontawesome}

% Tabularx package for custom tables
\usepackage{tabularx}

% Define navyblue color
\definecolor{navyblue}{RGB}{0,54,123}

%----------------------------------------------------------------------------------------
%   Customizations
%----------------------------------------------------------------------------------------

% Define italicitem, bolditem, and plainitem commands
\newcommand{\italicitem}[1]{\item{\textit{#1}}}
\newcommand{\bolditem}[1]{\item{\textbf{#1}}}
\newcommand{\plainitem}[1]{\item{#1}}

% Define user-friendly link command for hyperlinks
\newcommand{\link}[2]{{\href{#1}{#2}}}

\newcommand{\entry}[2]{#1 & #2 \tabularnewline} % Defines an entry with two arguments: #1 for the first column and #2 for the second column

%----------------------------------------------------------------------------------------
%   Define envsection command for defining a new environment section
%----------------------------------------------------------------------------------------

\newcommand{\tableEnv}[2]{%
  \begin{rSection}{#1} % Begin rSection with the given name
    \begin{adjustwidth}{0.0in}{0.1in} % Set the left and right margins
      \begin{tabularx}{\linewidth}{@{} >{\bfseries}l @{\hspace{6ex}} X @{}}
        #2 % Print the content inside the tabularx environment
      \end{tabularx}
    \end{adjustwidth}
  \end{rSection}
}

%----------------------------------------------------------------------------------------
%   Begin document
%----------------------------------------------------------------------------------------

% Set name with navyblue color
\name{\color{navyblue} Ram Narayan Singh }

\begin{document}

\printPersonalInfo{
  \personalInfo{\tag{Res}\info{ Behind Aparajita Aptt., B. H. Colony, K. Bagh, Patna - 800026}}
  \personalInfo{\tag{E-mail}\info{singhram914@gmail.com} \infoSeparator\tag{Mob}\info{+91-9431496830}}
  \personalInfo{Retired Teacher, Active in the field of Literature}
}

\tableEnv{Introduction}{
	{\normalfont A retired teacher deeply involved in literature, specializing in translations. } \\ 
	{\normalfont Proficient in English, Maithili, Hindi, Sanskrit, and Malayalam. }
}

%----------------------------------------------------------------------------------------
%   Work experience section
%----------------------------------------------------------------------------------------

\begin{rSection}{Notable Translated Publications}
	
	% First work experience entry
	\begin{rSubsection}{Malahin}{2011}{Sahitya Akademi, New Delhi}{Novel}
		\item Awarded with Sahitya Akademi award for translation in Maithili for 2014.
		\item Translation of Malayalam Novel ``Chemmeen" from the notable author Thakazhi Shivashankra Pillai.
	\end{rSubsection}
	
	% Second work experience entry
	\begin{rSubsection}{Poojaghar}{2016}{National Book Trust, India}{Drama Collection}
		\item Translation of three Malayalam dramas ``Snehdootan",  ``Poojamudi", ``Karutta Daivatte Tedi" to Maithili.
		
	\end{rSubsection}
	
	% Third work experience entry
	\begin{rSubsection}{Jagrata}{2009}{Bhasha Sangam Publications}{Short Story Collection}
		\item Translation of ``Chhatata Kohra" by Mithelesh Kumari Mishra from Hindi to Malayalam
		\item  Honored with ``Bhasha Samman" by Bhasha Sangam, Allahabad
	\end{rSubsection}
	
	% Fourth work experience entry
	\begin{rSubsection}{Sri Narayan Guru}{2021}{Sahitya Akademi, New Delhi}{Bibliography}
		\item Translation from Malayalam to Maithili of bibilography ``Sri Narayan Guru" by T. Bhaskaran
		
	\end{rSubsection}
	
\end{rSection}


%----------------------------------------------------------------------------------------
%   Education section
%----------------------------------------------------------------------------------------

    % Master's degree entry
    \tableEnv{Education}{
    	\entry{Ph. D}{Thesis Title: Critical Study of Rajashekhar's Literary Works (Sanskrit) }
    	\entry{Diploma}{Diploma in Malayalam from SRLC, CIIL, Mysuru}
    	\entry{M. A.}{Cleared with 1st division form Bihar University, Muzaffarpur (Sanskrit)}
    	\entry{M. Ed}{Completed in 1992 from Himachal Pradesh University, Shimla}
    }
    

%----------------------------------------------------------------------------------------
% Language proficiencies section
%----------------------------------------------------------------------------------------

\tableEnv{Language proficiencies}{
    \entry{English}{Can comfortably understand, read and write}
    \entry{Maithili}{Mother tongue}
    \entry{Malayalam}{Can comfortably understand, read and write}
    \entry{Sanskrit}{Fluent}
    \entry{Hindi}{Fluent}
}

\end{document}